\chapter{i3wm} \label{Chapi3wm}

i3wm is a tiling windows manager for Linux. The benefits of using tiling windows
managers in general is maximization of your screen real estate. Additionally,
i3wm uses Vim-like keybindings to enable you to navigate your machine almost
entirely using your keyboard. Here are some basic tips:
\begin{itemize}
    \item The official user guide can be found at \cite{stapelberg2015i3}
    \item Good Youtube Video on Configuring i3: \cite{codecast2015i3wm}
    \item To install: sudo apt-get install i3
    \item Use mod$+$d (where mod is set by default to be your Windows key) to open
        a search bar for your programs. For example, to run file explorer:
        mod$+$d then type "nautilus"
    \item To open Ubuntu settings, use "env XDG\_CURRENT\_DESKTOP$=$GNOME
        gnome-control-center". You'll find that you may need to Google many
        basic commands for navigating Ubuntu as many of the default icons are
        not available.
\end{itemize}

%==============================================================================
\section{Standard Configuration}
%==============================================================================
%------------------------------------------------------------------------------
\subsection{Generate Default Config File and Symlink}
%------------------------------------------------------------------------------
\begin{enumerate}
    \item You should be prompted to do so immediately upon first run of i3. However, you can also
        do so by using "i3-config-wizard" in the terminal.
    \item The config file is located in $\sim$/.config/i3/config
    \item To create a symlink, CUT, paste and rename $\sim$/.config/i3/config to your desired folder
        (say $\sim$/.vim/i3\_config)
        then use "ln -s $\sim$/.vim/i3\_config $\sim$/.config/i3/config"
    \item After making edits, use "mod+shift+r" to restart i3
\end{enumerate}

%------------------------------------------------------------------------------
\subsection{Configure i3 for Dual Monitors}
%------------------------------------------------------------------------------
\begin{enumerate}
    \item In terminal, use "xrandr" to see your connected monitors
    \item Then use "xrandr --output HDMI-1 --auto --left-of eDP-1" to set your HDMI-1 monitor to
        the left of your eDP-1 monitor
    \item To set it so that it starts up in this way, put "exec --no-startup-id xrandr --output
        HDMI-1 --left-of eDP-1 --auto" in the config file
\end{enumerate}

%------------------------------------------------------------------------------
\subsection{Wallpaper}
%------------------------------------------------------------------------------
\begin{enumerate}
    \item sudo apt-get install feh
    \item Add to config: "exec --no-startup-id feh --bg-fill
        $\sim$/.vim/Wallpapers/my\_wallpaper.jpg"
\end{enumerate}

%------------------------------------------------------------------------------
\subsection{Remove Title and Borders}
%------------------------------------------------------------------------------
Add the following to config file:
\begin{enumerate}
    \item new\_window pixel 0
    \item new\_float pixel 0
\end{enumerate}

%------------------------------------------------------------------------------
\subsection{Volume and Brightness Controls for Laptop}
%------------------------------------------------------------------------------
From \cite{quidproquo2018volume}:
\begin{enumerate}
    \item sudo apt-get install xbacklight alsa-utils pulseaudio
    \item Add the following to config:
        \begin{lstlisting}
# Pulse Audio Controls
bindsym XF86AudioRaiseVolume exec --no-startup-id pactl set-sink-volume 0 +5% # increase volume
bindsym XF86AudioLowerVolume exec --no-startup-id pactl set-sink-volume 0 -5% # decrease volume
bindsym XF86AudioMute exec --no-startup-id pactl set-sink-mute 0 toggle # mute volume

# Sreen brightness controls
bindsym XF86MonBrightnessUp exec xbacklight -inc 20 # increase screen brightness
bindsym XF86MonBrightnessDown exec xbacklight -dec 20 # decrease screen brightness
        \end{lstlisting}
\end{enumerate}

%------------------------------------------------------------------------------
\subsection{Screenshots}
%------------------------------------------------------------------------------
From \cite{brainlessdeveloper2017screenshot}:
\begin{enumerate}
    \item sudo apt-get install scrot
    \item sudo apt-get install xclip
    \item Add this to config file:
        \begin{itemize}
            \item bindsym --release Print exec scrot
                'screenshot\_\%Y\%m\%d\_\%H\%M\%S.png' -e 'mkdir -p ~/Downloads
                \&\& mv \$f $\sim$/Downloads \&\& xclip -selection clipboard -t
                image/png -i $\sim$/Downloads/`ls -1 -t $\sim$/Downloads | head -1`'
            \item bindsym --release Shift+Print exec scrot -s
                'screenshot\_\%Y\%m\%d\_\%H\%M\%S.png' -e 'mkdir -p
                $\sim$/Downloads \&\& mv \$f $\sim$/Downloads \&\& xclip
                -selection clipboard -t image/png -i $\sim$/Downloads/`ls -1 -t
                $\sim$/Downloads/          screenshots | head -1`'
        \end{itemize}
\end{enumerate}
Note that there is some strange issue on my machine where xclip takes up 99\%
CPU usage. Therefore, after each screenshot, you'll need to terminate xclip via
the gnome-system-manager.

%------------------------------------------------------------------------------
\subsection{Shutdown Options:}
%------------------------------------------------------------------------------
From \cite{jchamley2012how}:
\begin{enumerate}
    \item Add the following to your config file:
        \begin{lstlisting}
# Set shut down, restart and locking features
bindsym $mod+0 mode "$mode_system"
set $mode_system (l)ock, (e)xit, switch_(u)ser, (s)uspend, (h)ibernate, (r)eboot, (Shift+s)hutdown
mode "$mode_system" {
bindsym l exec --no-startup-id ~/.vim/i3_exit.sh lock, mode "default"
bindsym s exec --no-startup-id ~/.vim/i3_exit.sh suspend, mode "default"
bindsym u exec --no-startup-id ~/.vim/i3_exit.sh switch_user, mode "default"
bindsym e exec --no-startup-id ~/.vim/i3_exit.sh logout, mode "default"
bindsym h exec --no-startup-id ~/.vim/i3_exit.sh hibernate, mode "default"
bindsym r exec --no-startup-id ~/.vim/i3_exit.sh reboot, mode "default"
bindsym Shift+s exec --no-startup-id ~/.vim/i3_exit.sh shutdown, mode "default"
# exit system mode: "Enter" or "Escape"
bindsym Return mode "default"
bindsym Escape mode "default"
        \end{lstlisting}
    \item Create and make executable $\sim$/.vim/i3\_exit.sh with the following contents:
        \begin{lstlisting}
#!/bin/sh
lock() {
    i3lock
}
case "$1" in
    lock)
        lock
        ;;
    logout)
        i3-msg exit
        ;;
    suspend)
        lock && systemctl suspend
        ;;
    hibernate)
        lock && systemctl hibernate
        ;;
    reboot)
        systemctl reboot
        ;;
    shutdown)
        systemctl poweroff
        ;;
    *)
        echo "Usage: $0                                                      {lock|logout|suspend|hibernate|reboot|shutdown}"
        exit 2
esac

exit 0
        \end{lstlisting}
\end{enumerate}

%------------------------------------------------------------------------------
\subsection{Assigning Programs to Specific Windows}
%------------------------------------------------------------------------------
From \cite{codecast2015i3wm} at around 25:27.
To check the class of an application:
\begin{enumerate}
    \item Open terminal
    \item Use "xprop"
    \item With cursor now as a cross-hair, click on application
    \item WM\_CLASS should display on the terminal as the second value
    \item Add to config:
        assign [class="google-chrome"] \$workspace1
\end{enumerate}

%------------------------------------------------------------------------------
\subsection{i3 Status Bar}
%------------------------------------------------------------------------------
I think by default, installing i3 on Ubuntu gives version 2.11. To check
version, use "i3status --version" in terminal. If not installed at all, use
"sudo apt-get install i3status".\\

Note that the sample config file in the man pages uses stuff that doesn't exist
in version 2.11. I followed the config file \cite{sampaioveiga2020i3status} from
\cite{sampaioveiga2020i3medium} which also uses some options that do not exist
in version 2.11. I couldn't figure out how to upgrade but since I only want the
bare minimum, I didn't bother and just wrote a very minimal config file.\\

Following \cite{stapelberg2015i3status}:
\begin{itemize}
    \item create the directory ~/.config/i3status
    \item in your i3 config file add:
        \begin{lstlisting}
            bar {
                    status_command i3status --config ~/.config/i3status/config
            }
        \end{lstlisting}
    \item create i3status.conf somewhere (suppose in .vim folder)
    \item ln -s $\sim$/.vim/i3status.conf $\sim$/.config/i3status/config
    \item Edit the config then refresh i3 using mod+SHIFT+r
    \item If there is an error, type "i3status -c" in terminal and the error will be displayed
\end{itemize}

%------------------------------------------------------------------------------
\subsection{Setting Mouse Speed}
%------------------------------------------------------------------------------
Setting Mouse speed:
\begin{enumerate}
    \item "xinput list" in terminal to get list of connected devices. Find the id for mouse
    \item "xinput list-props \tlangle id\trangle" to get list of settings for the device.
    \item Find the bracketed number for "libinput Accel Speed (\tlangle number\trangle)"
    \item "xinput --set-prop \tlangle id \trangle \tlangle number \trangle
        \tlangle value \trangle" for value in [-1.0, 1.0]. This reduces the mouse speed
\end{enumerate}

%------------------------------------------------------------------------------
\subsection{Floats}
%------------------------------------------------------------------------------
To ensure that matplotlib opens as float, to your i3\_config, add:\\

for\_window [class="matplotlib"] floating enable\\

%==============================================================================
\section{i3 in Windows}
%==============================================================================
There are quite a few AutoHotkey repos for binding keys to switch between
desktops. However, many of them do not seem to work due to perhaps updates to
the OS. The one I found to work best can be found here:
\cite{pmb6tz2020windows}.\\

However, it doesn't seem to be the best idea to emulate i3 by switching between
desktops. The key difference I found is that you cannot assign separate monitors
a separate desktop; each desktop takes into account all monitors. For that
reason, I have simply opted to keep the default Win keybindings which correspond
to the task bar order. Instead, I have elected to implement the other i3wm
bindings such as Win-f to toggle maximize and Win-m to toggle moving windows
between screens.

%------------------------------------------------------------------------------
\subsection{Disabling Win+l}
%------------------------------------------------------------------------------
Since this is not exactly a windows tiling manager, I have also opted to use the
Win+arrow key functionality to organize my windows. However, I would like to do
so using the sacred Vim keys. A major issue is that Win+l locks the computer.
It's a bit of a hassle to disable this as you have to disable the whole computer
lock function, but it can be done using the following steps \cite{glenn2016disable}:
\begin{enumerate}
    \item Use Win+r to open the run app
    \item Type "regedit" and open
    \item Navigate to "HKEY\_CURRENT\_USER/SOFTWARE/Microsoft/Windows/CurrentVersion/Policies/"
    \item Right-click the "Policies" key and choose "New \trangle Key". Name the new
        key "System". Note that this may already be there. If so, this step can
        be skipped Right-click the "System" key and choose "New \trangle DWORD (32-bit)
        Value". Name this "DisableLockWorkstation"
    \item Double-click DisableLockWorkstation. Change the value from 0 to 1
\end{enumerate}
These changes should take place immediately without needing to restart. Now
anything bound to Win+l can be used. I should mention that in general this is
inadvisable as you are not only disabling the Win+l binding but your computer's
ability to lock itself (even after waking sleep mode).

%------------------------------------------------------------------------------
\subsection{Modifier Subtleties}
%------------------------------------------------------------------------------
Modifier keys refer to the control, alt, shift and windows keys.  The effect of
these modifiers persist through remappings. For example, the remap a::b would be
imply typing the capital A by either holding shift or having capslock on would
print the capital B. Further, holding control and pressing the `a' key would
output holding control and pressing the `b' key \cite{autohotkeys2005remap}.
Under the hood, the "::" remapping command uses the \emph{Blind} mode which
avoids releasing the modifier keys \cite{autohotkeys2005send}. That is, if the
user continues to hold down the modified key, the remapping program will not
logically register it as released. For example, the following remapping
\begin{lstlisting}
+s::Send {Blind}abc
\end{lstlisting}
would send ABC rather than abc because the user is holding down Shift symbolized
by the + symbol. Another example is the remapping
\begin{lstlisting}
^space::Send {Ctrl up}
\end{lstlisting}
automatically pushes control back down if the user is sill physically holding
control whereas
\begin{lstlisting}
^space::Send {Blind}{Ctrl up}
\end{lstlisting}
allows control to be logically up even though it is physically down.\\

Returning to the emulation of i3, suppose we want to map Win+Shift+h to Win+Left
which moves a window left. Since shift is held down, it will instead logically
register as move the window to the monitor to the left. Therefore, we require
the following mapping:
\begin{lstlisting}
#+h::
Send {Blind}{Shift up}{Left}{Shift down}
return
\end{lstlisting}
where the \# symbolizes the Win key and the + symbolizes the shift key. The last
\{Shift down \} allows for the use of Win+Shift+h,..h,h,h to continuously move
the windows without needed to repress Win+Shift each time.

%-----------------------------------------------------------------------------k
\subsection{Autostart AutoHotKey Script} \label{SecAutostart}
%------------------------------------------------------------------------------
There are two methods to setting your configuration to open on start up. I found
that on different machines, at least one of the two methods will work.\\

The first method is the simplest. Simply create a shortcut to your .ahk file
and place it in the\\
"C:/Users/Username/AppData/Roaming/Microsoft/Windows/Start
Menu/Programs/Startup" directory.\\

The second method is more complicated but, in general, more versatile. This
requres the use of the Task Scheduler \cite{bashkarla2016how}:
\begin{enumerate}
    \item Open the start menu and search "Task Scheduler"
    \item On the right panel, select "Create Basic Task" option
    \item Name the task anything you like and click next
    \item Select "When I log on" option and click next
    \item Select "Start a program" option and click next
    \item In the "Program/script" text box, enter your path to AutoHotkey.exe
        (i.e. "C:/Program Files/AutoHotkey/AutoHotkey.exe") with the quotation marks
    \item In the "Add arguments" text box, add the path to your .ahk file
        without the quotation marks
    \item Click Next then Finish
    \item To check if it works, find your newly added task in the "Task Scheduler Library" and
       click "Run"
\end{enumerate}

