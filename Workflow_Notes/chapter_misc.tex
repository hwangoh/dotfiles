\chapter{Miscellaneous} \label{ChapMisc}

%==============================================================================
\section{.bashrc Stuff}
%==============================================================================
%------------------------------------------------------------------------------
\subsection{Run \textbf{ls} after \textbf{cd}}
%------------------------------------------------------------------------------
Following `frabjous' answer to the StackOverflow question
\cite{robkohr2011make}, you can add the following to your .bashrc:
\begin{lstlisting}
function cd {
    builtin cd "$@" && ls -F
}
\end{lstlisting}
He also mentions that he adds
\begin{lstlisting}
    [ -z "$PS1" ] && return
\end{lstlisting}
before this so that ``everything after that line only applies to interactive
sessions, so this doesn't affect how \textbf{cd} behaves in scripts." How this
works is also explained by him:\\
``$[ -z "\$PS1" $] checks if the \$PS (interactive prompt variable) is `zero
length' (-z). If it is zero length, this means it has not been set, so Bash must
not be running in interactive mode. The \&\& return part exits from sourcing
.bashrc at this point, under these conditions."

%==============================================================================
\section{Ranger}
%==============================================================================
Following \cite{linuxcompendium2019ranger}:
\begin{enumerate}
    \item git clone https://github.com/hut/ranger.git
    \item cd ranger
    \item sudo make install
\end{enumerate}
To start ranger use "ranger".

%------------------------------------------------------------------------------
\subsection{Configuration}
%------------------------------------------------------------------------------
After the configuration directory has been created by the Ranger, you can now copy its configuration files by running the following commands in terminal:
\begin{itemize}
    \item "ranger --copy-config=all". Now you can run "cd ~/.config/ranger" to see the
        configuration files.
\end{itemize}
